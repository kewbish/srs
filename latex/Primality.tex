\documentclass{article}
\usepackage{style}
\usepackage[utf8]{inputenc}
\usepackage{lipsum}
\usepackage{placeins}
\usepackage{booktabs} 
% booktabs package lets you create beautiful tables
% use tablesgenerator.com and select the booktabs option

\graphicspath{{media/}}

\title[Primality Testing]{Primality Testing}
\author[Emilie Ma]{%
Emilie Ma\\%
University of British Columbia\\%
\texttt{kewbish@gmail.com}%
}
\date{\today}
\mentor{Pressiana Marinova\\%
Occado Technology Sofia\\%
\texttt{pressiana.marinova@gmail.com}%
}

\begin{document}
\maketitle
\newpage
\begin{abstract}
    Abstract
\end{abstract}

\section{Introduction}
Introduction

\section{Methods}

\subsection{Euler's Primality Test}
The results over three trials of Euler's Primality Test are shown in Table \ref{table:euler}. Each trial tested a different $k$ value, and consisted of:
\begin{itemize}
    \item{Generating a random set ($S$) of $10^4$ integers such that $10^6 < x < 2*10^6$}
    \item{Using SageMath's \texttt{is\_prime} to check for primality for each integer in $S$}
    \item{Running Euler's primality test with $k$ bases tried on each integer in $S$}
    \item{Counting all pseudoprimes which passed Euler's but not Sage's primality test}
    \item{Repeat for three sub-trials, average results and return lowest number of bases tried (lowest $k$) that returned the lowest number of pseudoprimes passed}
\end{itemize}
Each trial was timed with the Linux \texttt{time} command, recording the real, or total elapsed wall time, spent. 

\section{Results}

\FloatBarrier
\begin{table}[h]
\caption{Raw data for Euler's Primality Test Trials}
\begin{tabular}{@{}llll@{}}
\toprule
                                       & \multicolumn{3}{l}{All Random Bases}  \\ \midrule
                                       & Trial 1     & Trial 2    & Trial 3    \\
Running Time                           & 1m50.609s   & 2m4.602s   & 1m53.909s  \\
Lowest k required                      & 3           & 5          & 2          \\
Pseudoprimes passed at lowest k        & 0           & 0          & 0          \\
Range of lowest k required             & \multicolumn{3}{l}{3}                 \\
Range of number of pseudoprimes passed & \multicolumn{3}{l}{0}                 \\\midrule
                                       & \multicolumn{3}{l}{Base 2}            \\\midrule
                                       & Trial 1     & Trial 2    & Trial 3    \\
Running Time                           & 1m45.231s   & 1m41.760s  & 1m37.776s  \\
Lowest k required                      & 2           & 1          & 3          \\
Pseudoprimes passed at lowest k        & 0           & 0          & 0          \\
Range of lowest k required             & \multicolumn{3}{l}{2}                 \\
Range of number of pseudoprimes passed & \multicolumn{3}{l}{0}                 \\\midrule
                                       & \multicolumn{3}{l}{Base 3}            \\\midrule
                                       & Trial 1     & Trial 2    & Trial 3    \\
Running Time                           & 1m52.067s   & 1m39.540s  & 1m37.529s  \\
Lowest k required                      & 2           & 2          & 1          \\
Pseudoprimes passed at lowest k        & 0           & 0          & 0          \\
Range of lowest k required             & \multicolumn{3}{l}{1}                 \\
Range of number of pseudoprimes passed & \multicolumn{3}{l}{0}                 \\\midrule
                                       & \multicolumn{3}{l}{Base 5}            \\\midrule
                                       & Trial 1     & Trial 2    & Trial 3    \\
Running Time                           & 1m51.691s   & 1m42.775s  & 1m52.078s  \\
Lowest k required                      & 2           & 2          & 2          \\
Pseudoprimes passed at lowest k        & 0           & 0          & 0          \\
Range of lowest k required             & \multicolumn{3}{l}{0}                 \\
Range of number of pseudoprimes passed & \multicolumn{3}{l}{0}                 \\\midrule
                                       & \multicolumn{3}{l}{Base 2 and Base 3} \\\midrule
                                       & Trial 1     & Trial 2    & Trial 3    \\
Running Time                           & 3m32.639s   & 3m8.813s   & 3m3.992s   \\
Lowest k required                      & 1           & 1          & 1          \\
Pseudoprimes passed at lowest k        & 0           & 0          & 0          \\
Range of lowest k required             & \multicolumn{3}{l}{0}                 \\
Range of number of pseudoprimes passed & \multicolumn{3}{l}{0}                 \\
                                       & \multicolumn{3}{l}{Base 3 and Base 5} \\\midrule
                                       & Trial 1     & Trial 2    & Trial 3    \\
Running Time                           & 3m17.024s   & 2m59.818s  & 3m5.117s   \\
Lowest k required                      & 1           & 1          & 1          \\
Pseudoprimes passed at lowest k        & 0           & 0          & 0          \\
Range of lowest k required             & \multicolumn{3}{l}{0}                 \\
Range of number of pseudoprimes passed & \multicolumn{3}{l}{0}                 \\\midrule
                                       & \multicolumn{3}{l}{Base 2 and Base 5} \\\midrule
                                       & Trial 1     & Trial 2    & Trial 3    \\
Running Time                           & 3m5.117s    & 2m59.818s  & 19.840s    \\ % FINISH TMRW
Lowest k required                      & 1           & 1          & 2          \\
Pseudoprimes passed at lowest k        & 0           & 0          & 0          \\
Range of lowest k required             & \multicolumn{3}{l}{1}                 \\
Range of number of pseudoprimes passed & \multicolumn{3}{l}{0}                 \\ \bottomrule
\end{tabular}
\end{table}
\label{table:euler}
\FloatBarrier

\section{Discussion}
Discussion of results

\section{Conclusion}
Conclusion

\nocite{*}
\bibliographystyle{plainnat}
\bibliography{references}

\end{document}

